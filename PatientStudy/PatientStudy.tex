\documentclass[type=dr, dr=rernat, accentcolor=tud7b,colorbacktitle, bigchapter, openright, twoside, 12pt ]{tudthesis}
%\documentclass[11pt,twoside,a4paper]{article}
\usepackage[english]{babel} 
\usepackage[utf8]{inputenc}
\usepackage{graphicx}
\usepackage{pstricks}
\usepackage{psfrag}
\usepackage{enumerate}
\usepackage{float}
\usepackage{epsfig}
\usepackage{geometry}
\usepackage{subfigure}
\usepackage{rotating}
\usepackage{minitoc}
%\usepackage{appendix}

%%%% 1 1/2 facher Zeilenabstand:	
\usepackage{setspace}
\onehalfspacing




\begin{document}
\chapter{Patient Study}

\section{Materials and methods}

In this section input data will be presented as well as treatment planning parameters and procedures for SDRT and CiT. Finally, analysis method will be described.



\begin{table}[H]
  \centering
%   \footnotesize
  \caption{Respiratory motion in the direction of the largest motion component (SI) for all investigated patients.
  Furthermore the lung tumor 
  location (left lung (L) or right lung (R)) is stated next to the tumor volume.}
  \begin{tabular}{|c|c|c|c|c|}
    \hline\hline
    Patient no & Lesion number & Lesion location & Stage &
    peak-to-peak motion [mm]\\
    \hline
    1 & 1 & LSL & IIa & 4.8  \\
    \hline\hline
  \end{tabular}
  \label{tab:patdata}
\end{table}


% \bibliographystyle{apalike}
% \bibliography{../ref.bib}{}
% \bibliographystyle{plain}

\end{document}