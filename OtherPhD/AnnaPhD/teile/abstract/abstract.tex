% \documentclass[type=dr, dr=rernat, accentcolor=tud7b,colorbacktitle, bigchapter, openright, twoside, 12pt ]{tudthesis}
% \usepackage[english]{babel} 
% \usepackage[utf8]{inputenc}
% \usepackage{graphicx}
% \usepackage{pstricks}
% \usepackage{psfrag}
% \usepackage{enumerate}
% \usepackage{float}
% \usepackage{epsfig}
% %\usepackage{geometry}
% \usepackage{subfigure}
% \usepackage{rotating}
% \usepackage{minitoc}
% \usepackage{appendix}
% 
% \usepackage{tikz}
% 
% %%%% 1 1/2 facher Zeilenabstand:	
% \usepackage{setspace}
% \onehalfspacing


% \begin{document}


\section*{Summary}

One treatment modality for atrial fibrillation, the most common cardiac arrhythmia, is radiofrequency catheter ablation in which 
fibrotic tissue is created around the pulmonary veins. The procedure has major drawbacks, a new treatment modality would hence be beneficial.
This is the first \textit{in silico} study for the feasibility of a non-invasive treatment of atrial fibrillation with 
scanned ion beams, which are successfully used in cancer radiotherapy. 
In collaboration with the Mayo Clinic (Rochester, Minnesota, USA) the first comparison between different irradiation techniques to cardiac 
target volumes was carried out in this work. In the presented results it was shown that the dose deposition to organs at 
risk could be drastically reduced compared to a potential non-invasive treatment of atrial fibrillation with photons. 
As a result of the actively applied ion beam, interference effects were observed when irradiating the moving cardiac volumes. 
The motion influences of respiration and heartbeat were studied individually and the resulting displacement was examined. 
To achieve a homogenous dose deposition in the moving target volumes, motion mitigation techniques (gating and rescanning) 
were successfully applied. The results will be validated in animal experiments at GSI in 2014. 


\section*{Zusammenfassung}

Eine Behandlungsm\"oglichkeit f\"ur Vorhofflimmern, der verbreitetsten Art von Herzrhythmusst\"orungen, ist Radiofrequenzablation mit Hilfe von 
Kathetern, bei der Narbengewebe um die Pulmonarvenen erzeugt wird. Dieser Eingriff hat schwerwiegenden Nachteile, so dass eine neue  
Behandlungsm\"oglichkeit w\"unschenswert w\"are. Dies ist die erste \textit{in silico} Studie, die die 
Durchf\"uhrbarkeit einer nicht-invasive Behandlungsm\"oglichkeit f\"ur Vorhofflimmern mit gescannten Ionenstrahlen untersucht. Diese werden 
erfolgreich in der Strahlentherapie von Tumoren eingesetzt. 
In dieser Arbeit wurde in Kollaboration mit der Mayo Clinic (Rochester, Minnesota, USA) der erste Vergleich zwischen verschiedenen 
Bestrahlungsarten von Zielvolumina im Herzen durchgef\"uhrt. In den Ergebnissen konnte gezeigt werden, dass die Dosisbelastung der umliegenden 
Risikoorgane im Vergleich zu einer potentiellen nicht-invasiven Behandlung von Vorhofflimmern mit Photonen drastisch reduziert werden konnte. 
Aufgrund des aktiv applizierten Ionenstrahls wurden Interferenzeffekte bei der Bestrahlung bewegter Herzvolumina beobachtet. 
Der Bewegungseinfluss von Atmung und Herzschlag wurden unabh\"angig voneinander untersucht und die resultierende Auslenkung erforscht. 
Um eine homogene Dosis in den bewegten Zielvolumen zu erreichen wurden bewegungskompensierende Methoden (gating und rescanning) erfolgreich 
angewandt. Die Ergebnisse sollen in Tierstudien, die 2014 an der GSI durchgef\"uhrt werden, validiert werden.

% \end{document}


