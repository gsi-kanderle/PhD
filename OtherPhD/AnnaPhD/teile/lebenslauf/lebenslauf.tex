\documentclass[type=dr, dr=rernat, accentcolor=tud7b,colorbacktitle, bigchapter, openright, twoside, 12pt ]{tudthesis}
\usepackage[english]{babel} 
\usepackage[utf8]{inputenc}
\usepackage{graphicx}
\usepackage{pstricks}
\usepackage{psfrag}
\usepackage{enumerate}
\usepackage{float}
\usepackage{epsfig}
%\usepackage{geometry}
\usepackage{subfigure}
\usepackage{rotating}
\usepackage{minitoc}
\usepackage{appendix}

%%%% 1 1/2 facher Zeilenabstand:	
\usepackage{setspace}
\onehalfspacing


\begin{document}

\newpage
\chapter*{Lebenslauf}

\section*{Pers\"onliche Daten}
\begin{tabular}{p{.206\textwidth}p{.794\textwidth}}
  \hfill Name & Anna Constantinescu\\
  \hfill Geburtstag & 31. August 1984\\
  \hfill Geburtsort & Bukarest, Rum\"anien\\
  \hfill Nationalit\"at & Deutsch\\
\end{tabular}

\section*{Universit\"are Ausbildung}
\begin{tabular}{p{.206\textwidth}p{.794\textwidth}}
  \hfill seit 11/2010 & \textbf{TU Darmstadt} \\
  & Promotion (Durchf\"uhrung an der \textbf{GSI})\\
  \hfill 07/2007--07/2010 & \textbf{TU Darmstadt}\\
  & Master of Science in Physik\\
  & Titel der Masterarbeit: (Durchf\"uhrung an der \textbf{GSI})\\
  & Optimisation of the ion range adaptation method for moving tumour treatment with ion beam tracking \\
  \hfill 07/2007--06/2008 & \textbf{NTNU Trondheim}\\
  & Austausch im Rahmen des ERASMUS Programms\\
  \hfill 04/2004--07/2007 & \textbf{TU Darmstadt}\\
  & Bachelor of Science in Physik\\
  & Titel der Bachelorarbeit:\\
  & Untersuchung der Ablenkeigenschaften eines Dipolmagneten und Entwicklung eines Teilchen-Raytracer \\
\end{tabular}

\section*{Schulische Ausbildung}
\begin{tabular}{p{.206\textwidth}p{.794\textwidth}}
  \hfill 08/1995--06/2004 & \textbf{Gymnasium Gernsheim}, Gernsheim\\
  \hfill 08/1991--07/1995 & \textbf{Insel K\"uhkopf Schule}, Stockstadt\\
\end{tabular}

\end{document}