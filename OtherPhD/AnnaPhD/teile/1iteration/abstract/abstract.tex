\documentclass[type=dr, dr=rernat, accentcolor=tud7b,colorbacktitle, bigchapter, openright, twoside, 12pt ]{tudthesis}
\usepackage[english]{babel} 
\usepackage[utf8]{inputenc}
\usepackage{graphicx}
\usepackage{pstricks}
\usepackage{psfrag}
\usepackage{enumerate}
\usepackage{float}
\usepackage{epsfig}
%\usepackage{geometry}
\usepackage{subfigure}
\usepackage{rotating}
\usepackage{minitoc}
\usepackage{appendix}

\usepackage{tikz}

%%%% 1 1/2 facher Zeilenabstand:	
\usepackage{setspace}
\onehalfspacing


\begin{document}


\section*{Summary}

One treatment modality for atrial fibrillation, the most common cardiac arrhythmia, is radiofrequency catheter ablation in which 
fibrotic tissue is created around the pulmonary veins. The procedure is invasive, requires the sedation of the patient, has varying success 
rates and is associated with major complications. A new treatment modality would thus be beneficial. In previous studies with photon irradiation  
the electrical pathway of the heart could be succesfully changed in animal models. Based on the experience gained in cancer radiotherapy, 
where carbon ions offered a highly conformal treatment possibility, the here presented work studied the feasibility of a non-invasive 
treatment of atrial fibrillation with scanned carbon ions. Due to displacment of the cardiac target volumes on one hand because of the respiration 
of the patient and on the other hand because of the heartbeat, interference effects between the target volume motion and the actively applied 
ion beam were expected (interplay effect). Motion mitigation techniques were hence studied. For respiratory motion, the beam was applied in only a fraction of the 
motion cycle (gating), while the heartbeat influence was compensated for by irradiating the target volumes multiply times with a fraction of 
the planned dose (rescanning). This was studied in human data for a pulmonary vein irradiation, as well as in porcine data for an AV node irradiation. 
The latter was carried out in preparation for the planned animal experiments at GSI in 2014, where the findings shall be valitated. 
It resulted that all studied motion mitigation deliveries 
were adequate techniques for such an application, but that organ at risk doses needed to be carefully examined due to the location of the 
target volumes close to critical structures like the esophagus. 

\newpage

\section*{Zusammenfassung}

Eine Behandlungsm\"olichkeit f\"ur Vorhofflimmern, die verbreiteste Art von Herzrhythmusst\"orungen, ist Radiofrequenzablation mit Hilfe von 
Kathetern, bei der Narbengewebe um die Pulmonarvenen erzeugt wird. Dieser Eingriff ist invasiv, ben\"otigt die Narkotisierung der Patienten, 
zeigt schwankende Erfolgsquoten auf und wird mit schwerwiegenden Komplikationen in Verbindung gebracht. Eine neue Behandlungsm\"oglichkeit 
ist demnach w\"unschenswert. Vorhergehende Studien mit Photonen Bestrahlung haben gezeigt, dass es m\"oglich war das kardiale 
Reizleitungssystem in Tieren erfolgreich zu ver\"andern. Basierend auf der Erfahrung aus der Strahlentherapie, in der Kohlenstoffionen 
eine h\"ochst konforme Bestrahlung erm\"oglichten, hat die hier vorgestellte Arbeit die Durchf\"uhrbarkeit untersucht, Kohlenstoffionen als 
eine nicht-invasive Behandlungsm\"oglichkeit fuer Vorhofflimmern zu nutzen. Auf Grund der Bewegung der kardialen Zielvolumina, zum einen 
wegen der Atmung des Patienten und zum anderen wegen des Herzschlages, sind Interferenzeffekte zwischen der Ziel Bewegung und dem aktiv 
applizierten Ionenstrahl erwartet. Bewegungskompensierende Methoden wurden deshalb untersucht. Bei der Atmung wurde eine unterbrochenen 
Bestrahlung angewandt, bei der der Strahl nur in einem Teil des Bewegungszyklus angewandt wurde, w\"ahrend der Herzschlag mit Hilfe einer 
Mehrfachbestrahlung, bei der nur ein Teil der Dosis in einem Durchlauf deponiert wird, kompensiert wurde. Dies wurde sowohl in menschlichen 
Daten an Hand einer geplanten Pulmonarvenen Bestrahlung, als auch in Schweinedaten f\"ur eine AV-Knoten Bestrahlung durchgef\"uhrt. 
Letzteres wurde auch im Hinblick auf die geplanten Tierexperimente durchgef\"uhrt, die zur Validierung an der GSI in 2014 geplant sind.
Zusammenfassend l\"asst sich sagen, dass alle untersuchten Bestrahlungen adequate Techniken sind um solche eine Anwendung zu gew\"ahrleisten, 
allerdings die Bestrahlung auf Grund von naheliegenden Risikoorganen, wie etwa der Speiser\"ohre, schwierig ist und die Dosisdeposition in diesen 
Organen genauestens analysiert werden m\"ussen. 



\end{document}
