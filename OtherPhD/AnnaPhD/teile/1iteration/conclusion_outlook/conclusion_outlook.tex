\documentclass[type=dr, dr=rernat, accentcolor=tud7b,colorbacktitle, bigchapter, openright, twoside, 12pt ]{tudthesis}
\usepackage[english]{babel} 
\usepackage[utf8]{inputenc}
\usepackage{graphicx}
\usepackage{pstricks}
\usepackage{psfrag}
\usepackage{enumerate}
\usepackage{float}
\usepackage{epsfig}
%\usepackage{geometry}
\usepackage{subfigure}
\usepackage{rotating}
\usepackage{minitoc}
\usepackage{appendix}

%%%% 1 1/2 facher Zeilenabstand:	
\usepackage{setspace}
\onehalfspacing


\begin{document}

% \dominitoc
% % in big toc display only chapters and sections
% \setcounter{tocdepth}{1}
% \tableofcontents

\chapter{Conclusion and outlook}
% \minitoc

Particle therapy enables to deposit a high target dose while sparing the surrounding tissue. Actively applied particles allow moreover for a 
higly conformal irradiation without the need of patient specific hardware. Due to the successfull treatment outcomes an increasing number of 
therapy centers treat patients with scanned particle therapy. Up to now clinical applications of these centers have been restricted to radiotherapy 
treatments where the tumor showed no intrafractional displacements. This is due to the intereference effects which can otherwise be observed 
between the excisting target motion and the actively applied particle beam, leading to local over and under dosages (interplay effect) and hence to the need of 
motion mitigation techniques. Recently, pilot studies have been conducted which enabled the irradiation of tumors with small intrafractional 
motion. This is the first work to study the irradiation of intrafractionally moving, non-cancerous target volumes. Its aim is to investigate 
the feasibility of a non-invasive treatment for cardiac arrhythmias like atrial fibrillation by particel radiosurgery. The search for a new 
treatment modality for this condition is motivated by the currently excisting possibility of catheter ablation, an anatomical based treatment 
approach which has varying success rates and can lead to severe side effects. Studies for the potential of a non-invasive treatment with 
photon irradiation already exist. Due to the differing interaction mechanisms of photons with matter in comparison 
to ions, a better sparing of the surrounding tissue was expected for the here studied radiosurgery with ions. 
This could be demonstrated in this thesis, where carbon ion and photon delivery were compared on the same data sets, enabling a direct 
comparison. The difference in dose deposition to the organs at risk was significant and made a strong case for the usage of ions in cardiac 
radiosurgery.\newline  
\newline
Ablation via particle radiosurgery was studied for the pulmonary veins atria junction in human data separately for the underlying respiratory 
and heartbeat motion. These intrafractional motion types have a differing motion period and amplitude. While the respiration is a rather slow 
motion which causes the pulmonary vein junction to move in particular in the superior-inferior direction up to more than 2cm, the heartbeat 
displays a fast motion period causing a chaotic motion of the ablation site with an amplitude of up to 1cm. The interplay effet caused by 
respiration was hence found to be more pronounced than in case of heartbeat motion. In order to guarantee a robust treatment 
delivery, motion mitigation techniques have also been studied in case of heartbeat displacements. 
For the influence of the respiratory motion, the interrupted irradiation during a selected part of the motion cycle (gating) was studied and 
resulted to be an adequate technique. Nevertheless, gating always results in the prolongation of treatment time. Alternative methods could be
jet ventilation or apneic oxygenation, in which the patient is given artifical breathing and hence kept in steady respiratory phase. 
This would result in a shorter treatment time and would reduce the technical requirements for the application. 
In order to mitigate the influence of the heartbeat motion an averaging effect of different interplay patterns by scanning the same slice 
multiple times (rescanning) was applied. Also this motion mitigation technique resulted in a good dose coverage, already for small rescan numbers. 
This delivery would not prolong the treatment time. Since other rascanning techniques are nevertheless known to result in better treatment outcomes, like e.g. 
breath-sampled rescanning, where the rescans are distributed over the motion cycle, a similiar technique should be investigated in the 
future for the cardiac motion (ECG-sampled rescanning). In case of the studied porcine data, where the cardiac target volumes also displayed 
a chaotic motion but had a more shallow motion region in common, cardiac gating could furthermore be a potential application. This would 
require the application of a fast beam extraction modality for synchrotrons with radiofrequency knock-out exciters, which exist e.g. at the 
Heidelberg Ion Therapy Center \cite{Sch11} and are tested in first studieds at GSI. In case of the use of protons or other particles and hence 
the useage of cyclotrons, such a fast beam application could be achieved without any further hardware. In preparation for the planned 
animal experiments with pigs at GSI, which will be an experimental validation of the here found treatment planning results, rescanning 
was found to be a well suited technique to overcome the heartbeat motion influence in the atrioventricular node of swine. For the respiratory 
motion of the animals, an artifical breathing, similar to the above proposed, will be used. \newline
\newline
In all presented treatment planning results a 
physical dose of 25Gy was used. Biological effects leading to an relative biological effectiveness higher than one might occur, even though 
preliminary studies did not support this assumption. Also older photon studied suggested that 20Gy are sufficient to induce fibrotic tissue 
in the heart. Nevertheless higher doses than the here stated might be needed to create a complete electrophysiological block in the desired 
target area. The planned dose escalation studies in the animal models will offer valuable clues in respect to this question. 
The treatment planning results obtained from Mayo Clinic were obtained on contrast-enhanced CT scans, since the contrast between cardiac muscle 
and blood was not sufficient in native CT scans. A closer analysis on the resulting range uncertainties are needed. Potential 
experimental validations were suggested in order to test for potential range differences which might endanger critical structures. These 
need to be investigated and tested for suitability.\newline
\newline
In general, even though the feasibility of the studied motion mitigation techniques was shown, it also became obvious that the non-invasive 
treatment is challenging due to the amount of organs at risk which are in direct proximity of the desired target area. 
Intensity modulated particle therapy was hence needed in order to fall below dose-volume limits stated for these structures in radiosurgery. 
Besides the here studied potential ablation sites of the pulmonary veins junction and the atrioventricular node, other applications are also 
conceivable in the future. Cather ablation started to be also used for isolation of low-voltage areas in the ventricles of patients who 
suffered a myocardial infraction in order to prevent the formation of life-threatening ventricular tachycardias \cite{Til14} \cite{Mad14}. 
It has furthermore been shown that the underlying myocardial scar and border zones can be visualized in contrast-enhanced CT scans \cite{Tia14}. 
Treatment planning for this condition is hence potentially feasible. Due to the larger distance of the ventricles to many critical structures 
this delivery might even be easier achievable. The feasibility is planned to be tested in the upcoming animal experiments. 





\begin{thebibliography}{9999999}

\bibitem[Mad14]{Mad14}{Madhavan M, Lehmann HI, Swale MK, Johnson SB, Parker KD, Curley M, Hindricks G and Packer DL: Impact of a Novel Heated Saline Augmented Needle Tip Catheter on Improving Ablative Lesion Penetration and Eliminating Deep Tissue Conducting Channels in Canine Infarcts; Abstract (PO05-162); HRS Scientific Session 2014}
\bibitem[Sch11]{Sch11}{Schoemers C, Feldmeier E, Haberer T, Naumann J, Panse R and Peters A: Implementation of an intensity feedback-loop for an ion-therapy synchrotron; Prcoeedings of IPAC 2011; 2851-22853}
 \bibitem[Tia14]{Tia14}{Tian J, Jeudy J, Smith MF, Jimenez A, Yin X, Bruce PA, Lei P, Turgeman A, Abbo A, Shekhar R, Saba M, Shorofsky S, Dickfeld T: Three-Dimensional Contrast-Enhanced Multidetector CT for Anatomic, Dynamic, and Perfusion Characterization of Abnormal Myocardium To Guide Ventricular Tachycardia Ablations; Circ Arrhythm Electrophysiol.; 3; 2010}
 \bibitem[Til14]{Til14}{Tilz RR, Makimoto H, Lin T, Rillig A, Deiss S, Wissner E, Mathew S, Metzner A, Rausch P, Kuck KH, Ouyang F: Electrical isolation of a substrate after myocardial infarction: a novel ablation strategy for unmappable ventricular tachycardias--feasibility and clinical outcome; Europace; 2014}

 \end{thebibliography}

\end{document}