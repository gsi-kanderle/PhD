
\chapter{Conclusion and outlook}

This is the first work to study the irradiation of intrafractionally moving, non-cancerous target volumes with carbon ions. Its aim was to 
investigate the feasibility of a non-invasive treatment for cardiac arrhythmias like atrial fibrillation by 
particle radiosurgery.\newline
\newline
The search for a new treatment modality for this condition is motivated by the currently existing possibility of 
catheter ablation, which has varying success rates and can lead to severe side effects \cite{Cap05, Cap10, Jong05, Her13, Gai10, Med13}. 
Studies for the potential of a non-invasive treatment with photon irradiation already exist \cite{Sha10}. 
Due to differing interaction mechanisms of photons with matter in comparison to ions, a better sparing of the surrounding tissue was 
expected for the here studied radiosurgery with ions \cite{Ber12}. This could be demonstrated in this thesis, where carbon ion and photon delivery were 
compared on the same data sets. The difference in dose deposition to the organs at risk (OAR) was significant and made a strong case to use ions in 
cardiac radiosurgery.\newline  
\newline
Due to successful treatment outcomes in cancer radiotherapy an increasing number of therapy centers treat patients with scanned particle 
therapy \cite{PTCOG13}. Up to now clinical applications of these centers have been mostly conducted on tumors which showed no 
intrafractional displacements \cite{Loe13}. This is due to the otherwise occuring interference effects between the existing target motion and 
the actively applied particle beam, leading to local over and under dosages (interplay effect) and hence to the need of motion mitigation 
techniques \cite{Phi92, Ber08}.\newline
\newline
The here relevant intrafractional motion of the pulmonary veins atria junction was studied separately in human data for the underlying 
respiratory and heartbeat motion. These intrafractional motion types differ in motion period and amplitude. While the respiration is a rather 
slow motion which causes the pulmonary vein junction to move in particular in the superior-inferior direction up to more than 2cm, the heartbeat 
displays a fast motion period causing a fairly irregular motion of the ablation site with an amplitude of up to 1cm. The interplay effect 
caused by respiration was hence found to be more pronounced than in case of heartbeat motion. In order to guarantee a robust treatment 
delivery, motion mitigation techniques have also been studied in case of heartbeat displacements. 
For the influence of the respiratory motion, the interrupted irradiation during a selected part of the motion cycle (gating) \cite{Kub96} was 
studied and was found to be an adequate technique for a 30\% gating window around end exhale. Nevertheless, gating always results in a 
prolonged treatment time. Alternative methods could be jet ventilation \cite{Hof03} or apneic oxygenation \cite{RPTC12}, where the patient is 
kept in a steady respiratory phase and which would result in a shorter treatment time. 
In order to mitigate the influence of the heartbeat motion, an averaging effect of different interplay patterns by scanning the same slice 
multiple times (rescanning) \cite{Phi92} was applied. Also this motion mitigation technique resulted in a good dose coverage, already for small 
rescan numbers. This delivery would not prolong the treatment time. Other rescanning techniques are nevertheless known to result in an 
increased robustness while requiring fewer rescan numbers, like e.g. breath-sampled rescanning \cite{Sec09} or phase-controlled rescanning 
\cite{Fur07}, where the rescans are equally distributed over the motion cycle. Thus a similar technique should be investigated in the 
future for the cardiac motion (ECG-sampled rescanning). In case of the studied porcine data, where the cardiac target volumes also displayed 
an irregular motion but had a more shallow motion region in common, cardiac gating could furthermore be a potential application. This would 
require the application of a fast beam extraction modality for synchrotrons with radiofrequency knock-out exciters, which exist, e.g., at  
NIRS \cite{Nod96, Fur05} and HIT \cite{Schoe11} and are tested in first studies at GSI \cite{Ber09}. In preparation for the planned 
animal experiments with pigs at GSI, which will be an experimental validation of the here found treatment planning results, rescanning 
was found to be a well suited technique to overcome the heartbeat motion influence in the atrioventricular node of swine. For the breathing 
motion of the animals a respirator will be used. \newline
\newline
In all presented treatment planning results a physical dose of 25Gy was used. Biological effects leading to a relative biological 
effectiveness higher than one might occur, even though preliminary studies did not support this assumption and a RBE of 1.1 was found. 
It is expected that the RBE reaches a plateau for high single fraction doses \cite{Cara07}, in which case the physical dose could be 
considered proportional to the biological dose and the deposition in the organs at risk could be directly scaled to the desired value. 
Older photon studied suggested that 20Gy are sufficient to induce fibrotic tissue in the heart \cite{Faj70, Faj73}. 
Nevertheless higher doses than the here stated might be needed to create a complete electrophysiological block in the desired 
target area. The planned dose escalation studies in the animal models will offer valuable data in respect to this question. 
The treatment planning results were obtained on contrast-enhanced CT scans, since the contrast between cardiac muscle 
and blood was not sufficient in native CT scans. A closer analysis on the resulting range uncertainties is needed. Potential 
experimental validations like small dose depositions in silicon diodes \cite{Bent12} or PET probing beams \cite{Lin12} were suggested in order to 
test for potential range differences which might endanger critical structures. These need to be investigated and tested for suitability.\newline
\newline
In general it became obvious that a non-invasive 
treatment is challenging due to the amount of OAR which are in direct proximity of the desired target area. Intensity modulated particle 
therapy was hence needed in order to adhere to dose-volume limits stated for these structures \cite{RTOG0631, RTOG0915}. 
Furthermore a delivery with an ion beam gantry would be beneficial as this allows to choose suitable beam entry channels, 
enabling a sparing of the critical structures.\newline
\newline
Besides the here studied potential ablation sites of the pulmonary veins junction and the atrioventricular node, other applications are also 
conceivable in the future. Cather ablation started to be also used for isolation of low-voltage areas in the ventricles of patients who 
suffered a myocardial infarction in order to prevent the formation of life-threatening ventricular tachycardias \cite{Til14, Mad14}. 
It has furthermore been shown that the underlying myocardial scar and border zones can be visualized in contrast-enhanced CT scans \cite{Tia14}. 
Treatment planning for this condition is hence potentially feasible. Due to the larger distance of the ventricles to many critical structures 
this delivery might even be easier achievable. Nevertheless the displacements of ventricular target sites are expected to be larger then the 
here studied motion, so that further research on suitable motion mitigation techniques might be needed. 

