%\documentclass[type=dr, dr=rernat, accentcolor=tud7b,colorbacktitle, bigchapter, openright, twoside, 12pt ]{tudthesis}
\documentclass[11pt,twoside,a4paper]{article}
\usepackage[english]{babel} 
\usepackage[utf8]{inputenc}
\usepackage{graphicx}
\usepackage{pstricks}
\usepackage{psfrag}
\usepackage{enumerate}
\usepackage{float}
\usepackage{epsfig}
%\usepackage{geometry}
%\usepackage{subfigure}
\usepackage{rotating}
%\usepackage{minitoc}
%\usepackage{appendix}

%%%% 1 1/2 facher Zeilenabstand:	
\usepackage{setspace}
\onehalfspacing




\begin{document}



\section{Introduction}


Lung cancer is one of the leading medical problems worldwide with approximately 1.4 million deaths per year [1]. 
Surgery is usually the first choice in treating localized non-small cell lung cancer (NSCLC). However, in recent 
years stereotactic body-radiation therapy with photons (SBRT) showed very promising results, with high local control-rates of NSCLC \cite{ref1174}.

Cancer is one of the leading world cause of death, with 1.4 million deaths each year. Lung cancer has the highest death rate, where just in Germany 1.8 Million died in 2015.
Leading course of treatment used to be surgery, however radiotherapy showes promising results.




\bibliographystyle{unsrt}
\bibliography{ref}

\end{document}
