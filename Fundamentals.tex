\documentclass[type=dr, dr=rernat, accentcolor=tud7b,colorbacktitle, bigchapter, openright, twoside, 12pt ]{tudthesis}
\documentclass[11pt,twoside,a4paper]{article}
\usepackage[english]{babel} 
\usepackage[utf8]{inputenc}
\usepackage{graphicx}
\usepackage{pstricks}
\usepackage{psfrag}
\usepackage{enumerate}
\usepackage{float}
\usepackage{epsfig}
%\usepackage{geometry}
%\usepackage{subfigure}
\usepackage{rotating}
%\usepackage{minitoc}
%\usepackage{appendix}

%%%% 1 1/2 facher Zeilenabstand:	
\usepackage{setspace}
\onehalfspacing




\begin{document}
\section{Research background and Fundamentals}

\section{Radiotherapy}

Since the discovery of X-rays in 1895 the radiation has been used by physicans for treating patients. 
In the begining only superfical diseases could be cured, but as time and technology progressed X-ray tubes gained on voltage
and allowing treatment of deeper suited tumors.

The radiation from linear accelerator was first used in medicine in 1953. Because the beam is more collimated and energies are
higher than X-ray tube the cure rates improved tremendously. The next big milestone was introduction of computeres in the field
of radiotherapy. This led to better diagnostic tools, such as computed tomography scans (CT), magnetic resonance imaging (MRI) and
positron emission tomography (PET). With those tools the location of the tumor could be much better estimated and hence the physicans
could easier prescribe treatment. The potential of computers were afterwards exploited also in treatment planning with intensity modulated
radiation therapy (IMRT) which, together with diagnostic tools, provides an exact dose shaping in accordance to patient specific tumors.

In 1946 it was discovered that protons have preferable depth dose profile compared to photons and could be therefore used in cancer treatment.
First patient treatment was preformed in early 1950's at Lawrence Berkeley Laboratory, USA. 



\subsection{Photon Radiotherapy}

\subsection{Ion beam Radiotherapy}



\end{document}