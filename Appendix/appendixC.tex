\documentclass[type=dr, dr=rernat, accentcolor=tud7b,colorbacktitle, bigchapter, openright, twoside, 12pt ]{tudthesis}
%\documentclass[11pt,twoside,a4paper]{article}
\usepackage[english]{babel} 
\usepackage[utf8]{inputenc}
\usepackage{graphicx}
\usepackage{pstricks}
\usepackage{psfrag}
\usepackage{enumerate}
\usepackage{float}
\usepackage{epsfig}
\usepackage{geometry}
\usepackage{subfigure}
\usepackage{rotating}
\usepackage{minitoc}
\usepackage{multirow}
\usepackage{listings}
%\usepackage{appendix}

%%%% 1 1/2 facher Zeilenabstand:	
\usepackage{setspace}
\onehalfspacing

\begin{document}
 
\chapter{Appendix C}

\begin{table}[H]
  \centering
%   \footnotesize
  \caption{Dose constraints for various critical organs for 1, 2 and 3 fractions, denoted as respective numbers. Limits were used in SBRT and PT treatment planning. Data taken from \cite{Benedict2010}}
  \begin{tabular}{c|c|c|c|c|c|c|c|c}
 Patient  & heart & spinalcord & smallerairways & esophagus & trachea & aorta & lungl & lungr\\
 \hline \hline 
1 & 3.57 & 1.66 & 2.57 & 1.8 & 0.1 & 1.68 & 3.05 & 2.95 \\
2 & 1.96 & 0.52 & 6.5 & 0.92 & 0.11 & 1.31 & 3.88 & 0.7 \\
4 & 0.77 & 1.18 & 2.16 & 1.68 & 0 &0.48 & 3.62 & 0.6 \\
5 & 0.03 & 0.27 & 0.78 & 0.55 & 1.01 & 0.76 & 1.52 & 0.22 \\
6 & 0 &0.28 & 5.53 & 0.43 & 0.41 & 0.97 & 1.46 & 0.23 \\
7 & 0 &0.49 & 0 &1.12 & 0 &0 &0.7 & 0.09 \\
8 & 0.02 & 0.26 & 0.03 & 0.7 & 1.85 & 0.4 & 1.44 & 0.3 \\
9 & 0.06 & 0.25 & 0 &0.49 & 1.14 & 1.49 & 2.29 & 0.37 \\
10 & 0.08 & 1.69 & 0 &1.15 & 1.94 & 1.51 & 2.39 & 0.4 \\
11 & 2.48 & 2.27 & 2.83 & 4.08 & 0.06 & 2.43 & 2.9 & 0.9 \\
12 & 0.06 & 0.81 & 3.02 & 0.77 & 0 &1.08 & 1.88 & 0.49 \\
13 & 4.72 & 1.2 & 2.65 & 2.97 & 0 &3.45 & 5.4 & 5.15 \\
14 & 0 &0.27 & 0 &0.58 & 0.51 & 0.04 & 0.38 & 0.06 \\
15 & 1.49 & 0.66 & 3.8 & 1.18 & 0 &2.38 & 1.79 & 0.61 \\
16 & 0 &0.75 & 0.13 & 0.62 & 1.0 & 0.73 & 1.4 & 0.28 \\
18 & 1.34 & 0.19 & 6.1 & 1.22 & 0 &1.57 & 1.3 & 0.2 \\
19 & 0.07 & 0.7 & 6.08 & 1.07 & 1.81 & 1.65 & 2.88 & 0.69 \\
20 & 6.4 & 2.02 & 6.33 & 4.65 & 0.92 & 5.75 & 8.82 & 2.24 \\
21 & 4.08 & 1.57 & 6.45 & 2.83 & 0 &1.84 & 2.5 & 3.05 \\
\hline\hline
  
  \end{tabular}
  \label{tab:oarlimits}
\end{table}


\begin{table}[H]
  \centering
%   \footnotesize
  \caption{Dose constraints for various critical organs for 1, 2 and 3 fractions, denoted as respective numbers. Limits were used in SBRT and PT treatment planning. Data taken from \cite{Benedict2010}}
  \begin{tabular}{c|c|c|c|c|c|c|c}
   & Critical  & \multicolumn{3}{c}{Threshold dose (Gy)} & \multicolumn{3}{|c}{Maximum point dose(Gy)}  \\
  Organ & volume (cc) & 1 & 2 & 3 & 1 & 2 & 3 \\
   \hline
   heart & 15 & 16 & 22 & 24 & 30 & 32 & 38\\
spinal cord & 0.35 & 10 & 14 & 18 & 21.9 & 23 & 30\\
smaller airways & 0.5 & 12.4 & 13.3 & 18.9 & 23.1 & 21 & 33\\
esophagus & 5 & 11.9 & 15.4 & 17.7 & 25.2 & 19.5 & 35\\
trachea & 4 & 10.5 & 20.2 & 15 & 30 & 16.5 & 40\\
aorta & 10 & 31 & 37 & 39 & 45 & 47 & 53\\
stomach & 10 & 11.2 & 12.4 & 16.5 & 22.2 & 18 & 32\\
\hline\hline
  
  \end{tabular}
  \label{tab:oarlimits}
\end{table}
  

\bibliographystyle{apalike}
\bibliography{../ref.bib}{}
% \bibliographystyle{plain}

\end{document}
