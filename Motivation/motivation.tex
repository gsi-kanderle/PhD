\documentclass[type=dr, dr=rernat, accentcolor=tud7b,colorbacktitle, bigchapter, openright, twoside, 12pt ]{tudthesis}
%\documentclass[11pt,twoside,a4paper]{article}
\usepackage[english]{babel} 
\usepackage[utf8]{inputenc}
\usepackage{graphicx}
\usepackage{pstricks}
\usepackage{psfrag}
\usepackage{enumerate}
\usepackage{float}
\usepackage{epsfig}
\usepackage{geometry}
\usepackage{subfigure}
\usepackage{rotating}
\usepackage{minitoc}
% \usepackage{dominitoc}
\usepackage{multirow}
\usepackage{listings}
%\usepackage{appendix}
%\usepackage[breaklinks=true]{hyperref}
%\usepackage{breakcites}

%%%% 1 1/2 facher Zeilenabstand:	
\usepackage{setspace}
\onehalfspacing


\chapter*{Motivation}
\addcontentsline{toc}{chapter}{Motivation}



\newpage

\section*{Scope of this work}

This is the first \textit{in silico} survey to study the feasibility of a non-invasive treatment modality for atrial fibrillation with carbon ion 
radiosurgery \cite{Ber12}. Thereby the irradiation of ablation sites in the junction between pulmonary veins and atria will be studied in human data. 
Another potential ablation site for atrial fibrillation, the AV node, will be studied in porcine data sets. A single fraction 
dose of 25Gy will be applied on the cardiac target volumes. \newline

When intending to irradiate non-static targets interference effects between target motion and the actively applied ion 
beam cause local under and over dosages \cite{Phi92, Ber08}. In order to deposit a homogenous dose in the target area, motion mitigation 
techniques are needed. Different motion mitigation techniques will be used in this dissertation. The interrupted irradiation during a selected 
part of the motion cycle (gating) \cite{Kub96} as well as the repeated scanning of the same slice with a reduced dose so that averaging 
effects cause a homogenous dose deposition in the target volume (rescanning) \cite{Phi92} will be analyzed. 
% where the irradiation of target areas on and in the heart were studied 
% in order to determine the feasibility of a non-invasive treatment modality for atrial fibrillation with carbon ions. 
For the feasibility of a non-invasive treatment modality for atrial fibrillation with carbon ions, two independent 
motion influences have to be considered. On the one hand the heart beat, a fast but small amplitude motion, and on the other hand the 
respiration of the patient, a typically slow motion with big amplitude. Respiration and heartbeat gated CTs of human patients 
will be studied and the resulting treatment planning studies will be presented. In preparation for animal studies planned at GSI in the 
summer of 2014 treatment plan results for porcine data will also be investigated. \newline

The structure of this dissertation is as follows. Chapter 1 will give an overview over the physical and biological fundamentals of radiotherapy. 
Different radiotherapy applications will be presented and a special emphasis will be given on the treatment of moving targets. Furthermore, 
the cause of atrial fibrillation and its resulting risk factors will be presented. Currently existing therapies and the potential 
benefit of a non-invasive treatment modality will be discussed. In chapter 2 the influence of the respiratory motion on 
pulmonary veins ablation sites in humans will be discussed. The feasibility to use gating as motion mitigation technique for this case will 
be investigated. In chapter 3 the displacement of the pulmonary veins ablation site due to heartbeat will be examined in human data. 
Rescanning will be studied as motion mitigation technique for this motion component. In preparation for the planned animal experiments, 
which will be carried out as a first experimental feasibility study in the summer of 2014 at GSI, chapter 4 will discuss 
an AV node ablation with carbon ions in porcine data. The underlying heartbeat motion will be studied and rescanning as motion mitigation 
technique will be utilized. Discussion of the overall results will be given in chapter 5, while chapter 6 will conclude the findings and 
give a short outlook on future directions. 

\bibliographystyle{apalike}
\bibliography{../ref.bib}{}
% \bibliographystyle{plain}

\end{document}
