\documentclass[type=dr, dr=rernat, accentcolor=tud7b,colorbacktitle, bigchapter, openright, twoside, 12pt ]{tudthesis}
%\documentclass[11pt,twoside,a4paper]{article}
\usepackage[english]{babel} 
\usepackage[utf8]{inputenc}
\usepackage{graphicx}
\usepackage{pstricks}
\usepackage{psfrag}
\usepackage{enumerate}
\usepackage{float}
\usepackage{epsfig}
\usepackage{geometry}
\usepackage{subfigure}
\usepackage{rotating}
\usepackage{minitoc}
% \usepackage{dominitoc}
\usepackage{multirow}
\usepackage{listings}
%\usepackage{appendix}
%\usepackage[breaklinks=true]{hyperref}
%\usepackage{breakcites}

%%%% 1 1/2 facher Zeilenabstand:	
\usepackage{setspace}
\onehalfspacing


%chapter*{Motivation}
%\addcontentsline{toc}{chapter}{Motivation}
\begin{document}
 



\section*{Motivation}
\addcontentsline{toc}{chapter}{Motivation}
In 2013 every fourth death in Germany was due to cancer and approximately 45 000 deaths were from lung and bronchial cancer \cite{Destatis2015}. In the last 30 years, there was a 180\% increase of deaths due to lung and bronchial cancer for women \cite{Destatis2015}.
The standard course of treatment for lung cancer is surgery, chemotherapy, radiotherapy or a combination of these. Surgery is usually the first choice of treatment for early stages of lung cancer. 
In recent years, however, state of the art photon radiotherapy, called stereotactic body-radiation therapy (SBRT) showed
promising results for treating lung cancer \cite{Baumann2009, Greco2011}. SBRT delievers high doses (up to 24 Gy) in 1 - 5 fractions, therefore dose to critical structure must be carefully considered.

In the last twenty years, ion beam therapy has proven to be a promising alternative to photon radiotherapy. Higher tumor control rates and better dose conformity can be achieved with superior physical and biological ion properties, when compared to photons \cite{Tsujii2008,Durante2010}.
A recent review made by Kamada et al reported a high 3-year survival rate for carbon-ions (76.9\%) for treating lung cancer in single fraction, with no late treatment-related complications \cite{Kamada2016}. 
The treatment used passive beam scanning, where patient specific absorbers are used to conform the dose to the tumor. Active beam scanning, on the other hand, can provide even better dose shaping, which is essential in hypo-fractionated treatment. 
However, interaction between tumor and scanned beam motion, called interplay, can severely degrade dose distribution in patient. Therefore designated motion mitigation techniques must be used for successful treatment of lung cancer with active beam scanning \cite{Bert2008}.

Successful treatment of tumors in abdomen region (liver and pancreases tumors) with scanned ion beams has been already done at HIT, Heidelberg (Germany) and CNAO, Pavia (Italy) \cite{Habermehl2013, Rossi2016} and first lung cancer patients are being treated at NIRS, Chiba (Japan) \cite{Mori2016}.
Studies on impact of scanned ion beam on lung cancer treatment are thus warranted, so eligible patients can be identified.

In this thesis we will address this challenges of treating lung cancer patients with active beam scanning and 
make a comparison between SBRT and scanned carbon-ion therapy. Patients particularly suited for carbon-ion therapy will try to be identified for a possible future treatment at designated facilities in Marburg and Heidelberg. 



\newpage

\section*{Scope of this work}

This is the first in silico comparision between SBRT and active scanning carbon-ion for non-small cell lung cancer (NSCLC). A time-resolved (4D) doses will be studied on a large patient dataset, including patients with multiple metastases.

In order to create carbon-ion treatment plans and calculate 4D doses, contours have to be propagated from planning computed tomography (CT) to all motion states in 4D-CT. Additionally, motion between 4D-CT states has to be quantified. 
This will be achieved with deformable image registration (DIR). 
Furthermore, a designated tool for DIR quality assurance (DIRQA) will be developed. A verification of DIR and DIRQA has been done on a human lung 4D-CT and on pig cardiac 4D-CT dataset.

To show potential of scanned carbon-ions in handling NSCLC, treatment plans for 19 patients, which were actually treated with SBRT, will be calculated. Afterwards, static and 4D doses with and without motion
mitigation will be analyzed. Doses to targets and organs-at-risk (OAR) will be analyzed and compared between carbon-ions and SBRT.

A special investigation will be made into patients with multiple NSCLC metastases. A treatment planning software will be modified, in order to handle multiple targets in a single patient. 
Treatment plans for patients with multiple NSCLC metastases will be generated with two different optimization techniques to tackle range changes in moving targets. Comparison to SBRT will be made regarding target coverage and OAR doses.

The structure of this dissertation is as follows. Chapter 1 will present an overview of physical and biological fundamentals of radiotherapy. Photon and particle radiotherapy will be presented, with an emphasis on the treatment of moving targets.
Additionally, a description of lung cancer will be given. Chapter 2 will present tools to handle DIR and DIRQA and verification of these tools. In chapter 3, comparison between SBRT and carbon-ions will be investigated on lung cancer patients.
Treatment for patients with multiple metastases in lung will be investigated in chapter 4. Overall results will be discussed in chapter 5 and the thesis will be concluded in chapter 6.



\bibliographystyle{apalike}
\bibliography{../ref.bib}{}
% \bibliographystyle{plain}

\end{document}
