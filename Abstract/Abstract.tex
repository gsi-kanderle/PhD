\section*{Abstract}


Stereotactic body image guided radiation therapy (SBRT) shows good results for lung cancer treatment. 
However, complications can arise at the end or during the treatment. 
Better normal tissue sparing might be achieved with scanned carbon ion therapy (PT) and hence reduce the number of complications. 
Therefore an in silico trial was conducted to find potential advantages of PT in treating lung cancer. 
A study included patients that were treated with SBRT at Champalimaud Center for the Unknown, Lisbon (Portugal). 
PT plans were calculated on 4D-CTs with different breathing motion patterns simulated. A special investigation was made into patients with multiple lung disease sites, 
where two 4D optimization techniques were used. For successful simulations
deformable image registration was used and a tool to provide its quality assurance has been developed.

The results showed that target coverage was comparable in SBRT and PT, while PT delivered less dose to critical structures, including heart, spinal cord, esophagus, trachea and aorta. 
Additionally, motion was successfully mitigated with resanning. 



\section*{Zusammenfassung}

