\chapter*{Abstract}


Stereotactic body image guided radiation therapy (SBRT) shows excellent results for the local control of early stage lung cancer. 
However, not all patients are eligible for SBRT, and advanced stage treatment is less successful and often associated with severe side effects. 
Scanned carbon ion therapy (PT) can deliver more conformal dose likely benefiting these patient groups.

Therefore an \textit{in silico} trial was conducted on early and advanced stage patients to identify potential advantages of PT. 
The patients were treated with SBRT at Champalimaud Center for the Unknown, Lisbon (Portugal). PT plans were simulated on 4DCTs, 
and rescanning was investigated for motion mitigation in 4D-dose calculations. A dedicated strategy for 4D intensity modulated particle therapy (IMPT)
was developed and applied for advanced stage patients with multiple lesions. For clinically valid and reliable results the deformable 
image registration - necessary for 4D-dose calculation - a quality assurance tool was developed and applied in the study.

The results showed that target coverage was comparable in SBRT and PT, while PT delivered significantly lower doses to 
most critical structures, especially the heart, lungs, and esophagus. A highly complex case of advanced stage lung cancer could be treated
in a single fraction of 24~Gy with PT, while SBRT could not deliver the full ablative dose treatment due to an excessive heart dose.
The mean heart dose was reduced from 10~Gy to 0.8~Gy with PT for this specific patient.


\chapter*{Zusammenfassung}

Stereotaktische Radiochirurgie (SBRT) zeigt exzellente Ergebnisse f\"{u}r die lokale Kontrolle von 
Lungenkrebs im Fr\"{u}hstadium. Viele Patienten sind allerdings nicht f\"{u}r die SBRT geeignet, und die 
Behandlung von sp\"{a}teren Stadien f\"{u}hrt oft zu schweren Nebenwirkungen. Die Bestrahlung mit gescanntem 
Kohlenstoff (PT) erm\"{o}glicht eine konformere Dosisapplikation, wovon gerade diese Patientengruppen 
profitieren k\"{o}nnten.

Eine \textit{in silico} Studie an Lungenkrebspatienten in fr\"{u}hen und sp\"{a}ten Stadien wurde durchgef\"{u}hrt, 
um m\"{o}gliche Vorteile von PT zu untersuchen. Die Patienten wurden am Champalimaud Center for the 
Unknown, Lissabon (Portugal) mit SBRT behandelt. PT Pl\"{a}ne wurden auf 4DCTs simuliert und zur 
Bewegungskompensation wurde Rescanning in 4D-Dosisberechnungen untersucht. Eine dedizierte Strategie 
f\"{u}r 4D Intensit\"{a}ts-modulierte Partikeltherapie (IMPT) wurde entwickelt und f\"{u}r Patienten im fortgeschrittenem 
Stadium mit mehreren L\"{a}sionen eingesetzt. F\"{u}r klinisch valide und verl\"{a}ssliche Ergebnisse wurde f\"{u}r Nicht-rigide
Bildregistrierung - f\"{u}r die 4D-Dosisberechnung unerl\"{a}sslich eine Strategie zur Validierung und Qualit\"{a}tssicherung entwickelt.

Es ergab sich eine vergleichbare Dosisabdeckung der Ziele f\"{u}r PT und SBRT, mit PT konnte die 
Dosisbelastung fast aller Risikoorgane aber signifikant gesenkt werden, insbesondere des Herzens, der Lunge und der Speiser\"{o}hre. 
In einem besonders komplexen Fall von Stufe IV Lungenkrebs konnte PT alle 5 L\"{a}sionen mit der vollen Dosis von 24~Gy abdecken, w\"{a}hrend dies mit SBRT durch die zu hohe 
Herzdosis nicht m\"{o}glich war – die mittlere Herzdosis konnte mit PT trotz voller Zieldosis von 10~Gy auf 0.8~Gy reduziert werden.
% w\"{a}hrend SBRT 
% durch die zu hohe Herzdosis von 25 Gy auf 22 Gy und weniger beschr\"{a}nkt war  die Herzdosis mit PT bei voller Zieldosis lag bei XX Gy.


