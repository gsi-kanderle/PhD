\section*{Abstract}


Stereotactic body image guided radiation therapy (SBRT) shows good results for lung cancer treatment. However, complications can arise at the end or during the treatment. 
Better normal tissue sparing might be achieved with scanned carbon ion therapy (PT) and hence reduce the number of complications. 
Therefore an in silico trial was conducted to find potential advantages of PT in treating lung cancer. 
A study was conducted on patients that were treated with SBRT at Champalimaud Center for the Unknown, Lisbon (Portugal). 
PT plans were calculated on 4D-CTs with different breathing motion patterns simulated. For successful simulation
deformable image registration was used and a tool to provide its quality assurance has been developed. 
The results of the study showed that target coverage was the same in SBRT and PT, while PT delivered less dose to OARs. 
Patients with large target volumes and with multiple disease sites would especially benefit from PT.



\section*{Zusammenfassung}

