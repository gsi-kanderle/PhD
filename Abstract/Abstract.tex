\section*{Abstract}


Stereotactic body image guided radiation therapy (SBRT) shows good results for lung cancer treatment. However, complications can arise at the end or during the treatment. 
Better normal tissue sparing might be achieved with scanned carbon ion therapy (PT) and hence reduce the number of complications. 
Therefore an in silico trial was conducted to find potential advantages of PT in treating lung cancer. 
A study was conducted on patients that were treated with SBRT at Champalimaud Center for the Unknown, Lisbon (Portugal). 
PT plans were calculated on 4D-CTs with different breathing motion patterns simulated. For successful simulation
deformable image registration was used and a tool to provide its quality assurance has been developed. 
The results of the study showed that target coverage was the same in SBRT and PT, while PT delivered less dose to OARs. 
Patients with large target volumes and with multiple disease sites would especially benefit from PT.




\section*{Zusammenfassung}

% Stereotaktische Radiochirurgie (SBRT) zeigt exzellente Ergebnisse f�r die lokale Kontrolle von Lungenkrebs im Fr�hstadium. 
% Viele Patienten sind allerdings nicht f�r die SBRT geeignet, und die Behandlung von sp�teren Stadien f�hrt 
% oft zu schweren Nebenwirkungen. Die Bestrahlung mit gescanntem Kohlenstoff (PT) erm�glicht eine konformere Dosisapplikation, 
% wovon gerade diese Patientengruppen profitieren k�nnten.
% 
% Eine in silico Studie an Lungenkrebspatienten in fr�hen und sp�ten Stadien wurde durchgef�hrt, um m�gliche Vorteile von PT zu untersuchen. 
% Die Patienten wurden am Champalimaud Center for the Unknown, Lisbon (Portugal) mit SBRT behandelt. PT Pl�ne wurden auf 4DCTs simuliert und 
% zur Bewegungskompensation wurde Rescanning in 4D-Dosisberechnungen untersucht. Eine dedizierte Strategie f�r 4D Intensit�ts-modulierte Partikeltherapie (IMPT) 
% wurde entwickelt und f�r Patienten im fortgeschrittenem Stadium mit mehreren L�sionen eingesetzt. F�r klinisch valide und verl�ssliche Ergebnisse wurde 
% f�r Nicht-rigide Bildregistrierung - f�r die 4D-Dosisberechnung unerl�sslich ? eine Strategie zur Validierung und Qualit�tssicherung entwickelt.
% 
% Es ergab sich eine vergleichbare Dosisabdeckung der Ziele f�r PT und SBRT, mit PT konnte die Dosisbelastung fast aller Risikoorgane 
% aber signifikant gesenkt werden, insbesondere des Herzens, der Lunge und der Speiser�hre. In einem besonders komplexen Fall von Stufe 
% IV Lungenkrebs konnte PT alle 5 L�sionen mit der vollen Dosis von 24 Gy abdecken, w�hrend SBRT durch die zu hohe Herzdosis von XX Gy 
% auf 22 Gy und weniger beschr�nkt war ? die Herzdosis mit PT bei voller Zieldosis lag bei XX Gy.

