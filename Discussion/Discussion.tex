\documentclass[type=dr, dr=rernat, accentcolor=tud7b,colorbacktitle, bigchapter, openright, twoside, 12pt ]{tudthesis}
%\documentclass[11pt,twoside,a4paper]{article}
\usepackage[english]{babel} 
\usepackage[utf8]{inputenc}
\usepackage{graphicx}
\usepackage{pstricks}
\usepackage{psfrag}
\usepackage{enumerate}
\usepackage{float}
\usepackage{epsfig}
\usepackage{geometry}
\usepackage{subfigure}
\usepackage{rotating}
\usepackage{minitoc}
% \usepackage{dominitoc}
\usepackage{multirow}
\usepackage{listings}
%\usepackage{appendix}
%\usepackage[breaklinks=true]{hyperref}
%\usepackage{breakcites}

%%%% 1 1/2 facher Zeilenabstand:	
\usepackage{setspace}
\onehalfspacing




\begin{document}
\chapter{Discussion}



DIR for adaptive radiotherapy - patient specific is neccesary, large DVF large errors %http://www.ncbi.nlm.nih.gov/pmc/articles/PMC4041490/

DIR between different modalities - PET/CT %http://www.tandfonline.com/doi/abs/10.3109/0284186X.2013.813964

RBE

24Gy(RBE) %http://www.sciencedirect.com/science/article/pii/S0360301615005179

Hypo-fractionation of liver cancer %http://ro-journal.biomedcentral.com/articles/10.1186/1748-717X-8-59  http://jco.ascopubs.org/content/early/2015/12/11/JCO.2015.64.2710.short

Future SBRT - 4pi %https://www.aapm.org/meetings/2014SS/documents/FUTUREOFSBRT2014Kupelian.pdf
SBRT in lung cancer will become standard 

Overview of lung cancer %http://www.tandfonline.com/doi/pdf/10.3109/0284186X.2011.590148

Reoccurent lung cancer %http://www.ncbi.nlm.nih.gov/pubmed/24016675

Heart and esophagus dose good predictors of mortality RTOG 0917 median survival 19.5 months vs 29.4 months with protons (http://www.ncbi.nlm.nih.gov/pubmed/24864278) for 74 Gy, but was 29 for 60 Gy with photons as well.

Photons vs Protons for stage III NSCLC RTOG 1308

Stephen Bowen - deliever more dose to damaged lung instead of functionated one

bio-adaptive particle therapy

SBRT is the lowest for cost per quality of life for early stage, but Protons beams scanning for advanced %http://www.ncbi.nlm.nih.gov/pubmed/26828647
SBRT limited: large tumors (>5 cm, more tumors, patients with poor lung funtction)

\bibliographystyle{apalike}
\bibliography{../ref.bib}{}
% \bibliographystyle{plain}

\end{document}