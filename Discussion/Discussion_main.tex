\chapter{Discussion}

This is a first in silico study directly comparing clinical stereotactic body radiation therapy (SBRT) with scanned carbon-ions (PT) for non-small cell lung cancer (NSCLC). 
Our results show that PT could be considered an alternative to SBRT, with the same tumor coverage and less dose to OARs. Furthermore, the study was expanded to patients with multiple
NSCLC disease sites. With state of the art 4D optimization, intensity modulated particle therapy (IMPT) was able to generate treatment plans with less OAR dose and comparable target coverage
to SBRT. It was possible to generate a single fraction plan with IMPT for a specific patient, where SBRT was limited due to high OAR dose.

Treatment of NSCLC with PT is influenced by interplay between tumor motion and beam scanning. It was shown that rescanning offers an
adequate motion mitigation. 

PT offers a precise dose shaping and it can be thus more prone to uncertainties. Calculation of time-resolved (4D) doses can be significantly
affected by errors in deformable image registration (DIR) \cite{Heath2006}. Special tools were developed in the scope of this thesis to ensure DIR quality assurance (DIRQA).
Tools were tested on a large dataset to ensure their validity.


\section{Deformable image registration and validation}

A single DIR algorithm was used in this study, B-Spline. In contrast to Demons algorithm, B-Spline should handle large deformations present in lung and cardiac 4D-CT well \cite{Tang2013}.
The DIR for lung 4D-CT had only a few inconsistencies and can be considered successful. On the other hand, the results suggest that B-Spline is inadequate for pig cardiac 4D-CT. The parameters used
in B-Spline DIR were similar in both cases of DIR. This could be improved by investigating the effect of parameters on DIR quality. 

The advantage of using open-source software for DIR, as explained in Section~\ref{RegistrationImplement} is that different DIR algorithms, 
optimization metrics and image types can be used. They can be accessed either with using
existing libraries, such as ITK \cite{Yoo2002}, or by writing designated software \cite{Fedorov2015}. 


In this study, DIR was used in contour propagation, 4D optimization and 4D dose calculation. The 4D optimization and 4D dose calculation require 
accurate DIR in each voxel. We ensured this by calculating DIRQA voxel-wise. DIR could also be tested with designated 4D dose calculation. Without 4D optimization, 
the treatment plan is optimized on a single phase 4D-CT. The DIR is then used in a 4D dose calculation by propagating dose from all voxels in each phase to the reference phase.
\textbf{?} 4D dose can also be calculated by applying separate plans to each 4D-CT phase and the results should be comparable.

Tests in DIRQA module can be divided into two groups - qualitative and quantitative. Qualitative tests are false color and checkerboard. They provide clear overview of the DIR result.
However they do not give any about the vector field quality. An example of the disadvantage of the qualitative test could be seen in pig cardiac 4D-CT, where qualitative test did not show
any errors in DIR, but errors were observed in vector fields. 

The quantitative tests used in DIRQA module are landmark distance, absolute difference, Jacobian and inverse consistency error (ICE). Absolute difference, Jacobian and ICE have undergone an extensive testing. The results suggest
that absolute difference gives us the least information about DIRQA, apart that it has to be lower after the DIR. 
We have shown that bigger deformations yield more deviations in Jacobian and ICE, which was also found by Stanley et al \cite{Stanley2013}. 
Furthermore, we have confirmed that Jacobian should always be positive for a successful DIR \cite{Rey2002}. Additionally, our results show that ICE should 
be smaller than maximum vector field magnitudes. Any deviations from mentioned trends should be thoroughly examined.

There are additional vector fields validation methods beside Jacobian and ICE, such as vector field curl \cite{Schreibmann2012}, unbalanced energy \cite{Zhong2007}, 
permutation and analysis of variance (ANOVA) tests \cite{Klein2009}.
It was demonstrated in a study by Salguero et al \cite{Salguero2011} that DIR errors greater than 1 mm can lead to large dose errors in high-dose gradient regions. 
Therefore the DIR accuracy has to be quantified at each image voxel in the high-dose 
gradient regions. In our study a focus was given on a complete registration to find potential errors. However, in future studies the regions of interest used
should be around the target, where high-dose gradients can occur. Furthermore the effect of image and vector field downsampling on DIRQA should be assessed.

%A solution to evaluate at each specific voxel and for patient-specific registration was given by Stanley at al \cite{Stanley2013}. 
%They proposed a computational phantoms and their deformations with a finite element module framework. 

Due to the lack of landmarks in all 4D-CTs landmark distance was not included in verification. Two contour based validation, dice similarity coefficient \cite{Varadhan2013} and Hausdorff distance \cite{Huttenlocher1993}
are planned to be implemented in DIRQA module. In literature many different attempts have been done to asses DIRQA with landmarks or contours. 
A study by Hardcastle et al \cite{Hardcastle2012} 
compared demons and Salient-Feature-Based registration with dice coefficient between propagated and physician drawn contours.
A multi-institutional study by Brock et al \cite{Brock2010} compared differences in propagated and oncologist drawn landmarks. 
A method has been developed by Castillo et al \cite{Castillo2009} to automatically identify landmark points
in lung patients images. However, visual based evaluations are limited in regions of uniform image intensity and by the number of the objects being tracked \cite{Kashani2008, Liu2012}.



\newpage
\section{Radiation treatment for non-small cell lung cancer}

\subsection{Non-small cell lung cancer in early stages}

Surgery is the gold standard for treating NSCLC in the early stages \cite{Roesch2014}. 
In recent years, however, SBRT showed similar results as surgery and SBRT is recommended for all high-risk surgical patients. A recent comparison study by Yu et al \cite{Yu2015} showed that
SBRT compared to surgery had lower intermediate mortality and toxicity. However, patients with long life expectancies were found to benefit more from surgery. 

There are several clinical scenarios where the use of SBRT might be limited. 
This includes treatment of centrally located tumors, tumors close to the chest wall, large tumors (radius > 5 cm) and 
multiple primary tumors \cite{Timmerman2006, Georg2008, Westover2012}.
The limitation of SBRT could open possibilities to other treatment modalities, such as particle therapy. 
Two of the mentioned scenarios, large tumors and multiple tumors, would benefit the most from particle therapy, according to the results shown
in this thesis and to the results published by Kadoya et al \cite{Kadoya2010}. In a study done at Francis H. Burr Proton Therapy Center 
patients who could not be treated with SBRT, due to the scenarios mentioned, were treated with passive proton beam in 3 - 5 fractions, 
delivering 42 - 50 Gy \cite{Westover2012}. They observed similar tumor local control rates as in SBRT (100\% in a two year follow-up) with limited toxicities. 
It should be stressed, that this were patients were rejected from SBRT treatment due to complexity and regardless proton therapy achieved similar results
to SBRT.


\subsection{Non-small cell lung cancer in advanced stages}

Treatment for early stage NSCLC is well established, however, more than 75\% NSCLC cases present themselves in an advance stage \cite{Jemal2009}, 
usually due to the lack of detection in the early stages. The standard of care for advanced NSCLC is concurrent chemotherapy \cite{Oshiro2014}.
Dose escalation studies showed favorable prognosis for doses higher than 70 Gy \cite{Hayman2001, Rosenman2002, Socinski2008}. 
The results of a recent phase 3 randomized trial by Bradley et al \cite{Bradley2015}, however, showed better survival rates for patients administered 60 Gy,
instead of 74 Gy. It was speculated that higher doses to heart and esophagus might have contributed to higher mortality rates \cite{Cox2012}. 
Results presented in this thesis show that mean dose to heart and esophagus would be on average 1 Gy
smaller with PT than with SBRT. For patients with multiple disease sites the the average mean heart and esophagus dose would be 4 and 3 Gy smaller.
Similar results were observed when comparing protons to SBRT \cite{Georg2008}. 

In a recent phase II study by Iyengar et al \cite{Iyengar2014} they treated stage IV NSCLC with SBRT and chemotherapy. 
They have irradiated 52 targets in 24 patients, 16 of them had more than one target. The results were promising, with 20 months median overall survival, 
compared to 9 months when treating with chemotherapy only \cite{Tsao2008}. Results in this thesis show, that patients with multiple disease sites 
would especially benefit from PT. Based on the poor prognosis that stage IV NSCLC patients have and on the results published by Iyengar et at,
stage IV NSCLC patients could be eligible candidates for PT treatment. Additionally, such patients usually exhibit chronic obstructive pulmonary disease and 
less dose to the lung is warranted \cite{Westover2012}. This further supports our claim, since our study showed substantial differences in 
doses to ipsilateral lung ($V_{20\%}$ was on average 15\% smaller in PT for patients with multiple disease sites) and 
contralateral lung as well - 70\% of patients did not receive any dose to contralateral lung, whereas SBRT deposited dose in contralateral lung in all patients.
The PT treatment planning for patients with complex geometry has to include 4D optimization and dose calculation as shown within this thesis. 
A special consideration has to be taken in dose to OAR limits, since they can be breached under different motion patterens.
Beam tracking \cite{Bert2007} or jet-ventilation \cite{Santiago2013} would be possible solutions, however, the former is not yet clinically avaliable and the latter 
significantly complicates treatment.

The results of a multi-institutional randomized trial, RTOG1308 \cite{RTOG1308}, comparing photons and particle therapy in treating NSCLC,
will have an important impact on treating NSCLC. The trial started in 2014.

\subsection{Motion mitigation}

While tumor motion influences photon treatment, it can be mittigated with proper margins \cite{Zou2014}. 
On the other hand, effects can be substantial when treating moving targets with scanned particle therapy \cite{Bert2008}.
It was shown in this thesis that rescanning is an adequate motion mitigation technique. However, rescanning has a degree of uncertanity, 
especially regarding maximum allowed point dose to OAR.
In hypofractionated treatment this limits are strict and exact dose to OAR must be known. 
A possible solution would be to simulate rescanning and 4D delivery in optimization process itself. 
However such solution is not yet feasible due to the complexity of the problem.

Comonnly used motion mitigation technique in photon and particle treatment is gating. While it provides less motion-induced dose errors, it prolonges treatment time.
A recent study by Zhang et al \cite{Zhang2015} included 
different breathing patterns, obtained from a MRI, on a 4D-CT and calculated 4D doses for liver cancer patients. They have shown that a gating window of 3 mm can result
in a 10\% efficiency of a duty-cycle, 
substantially prolonging treatment. Additionally, they have shown that neither volumetric or slice-by-slice rescanning could achieve good target coverage.
However, good target coverage was obtained with combination
of gating and rescanning. Their results suggest that a combination of gating and rescanning would currently be the best solution for treating NSCLC patients with PT.

Between rescanning, gating and beam tracking is beam tracking the most precise technique, since it requires no internal margins for target \cite{Bert2011}. 
Current clinical implementations of tracking in photon radiotherapy \cite{Kilby2010, Keall2014} can not be directly used in particle therapy, 
since they only provide position of single internal points. Fassi et al 
\cite{Fassi2015} were able to account for inter- and intra-fractional variability of patient's anatomical configuration with a designated modeling technique \cite{Fassi2014}.
The measured median of water-equivalent path length in target was within 2 mm of a simulated one. For actual clinical impementation it will be necesarry to test the model
on a large patient dataset.


\subsection{Outlook}

Recent advances in photon radiotherapy allow the usage of noncoplanar beams, a so-called 4$\pi$ optimization \cite{Dong2013}. In a recent study, Dong et al \cite{Dong2013b} showed that
4$\pi$ yielded better target coverage and OAR sparring than SBRT for NSCLC patients. 
They have reported reduction of $D_{Max}$ in heart, esophagus and spinal cord by 32\%, 72\% and 53\%, respectively, showing
the potential of a 4$\pi$ optimization. According to this thesis, PT is able to reduce the $D_{Max}$ even further, with a reduction of 57\%, 87\% and 83\% for heart, esophagus and spinal cord,
respectively. The numbers, however, should be compared with caution, since they were obtained from a different set of patients. 
A future study, directly comparing SBRT, $4\pi$ and PT for NSCLC is thus warranted.

Robust optimization seems to be gaining on popularity for PT. Standard margin defenition to account for uncertainties fails short in PT, while the inclusion of uncertainties in 
optimization process can substantially improve treatment plans \cite{Chen2012}. The robstuness optimization is now possible even for a 4D optimization \cite{Liu2016}, opening a wide
field of new possibilities. 
\newpage

\chapter{Conslusion}

In this work, designated tools were developed to handle deformable image registrations on different image sets. Furthermore, several test were 
integrated to ensure quality assurance of the deformable image registration. The tools developed underwent an extensive testing on a large patient dataset
and were able to produce deformable image registrations as well as find errors in it.

The deformable image registration was than used for 4D dose calculations of scanned carbon-ions treatment plans in lung cancer patients. The results were compared
to state of the art photon treatment plans. With rescanning as a motion mitigation technique, carbon-ions were able to achieve the same target coverage as
photon plans, while reducing the dose to critical structures including heart, spinal cord, esophagus, trachea and aorta.

For patients with multiple disease sites, a treatment planning system was modified to be able to create plans for such patients. 
Treatment plans for patients with multiple lung disease sites were generated with two recent 4D optimization algorithms. Both provided
comparable target coverage to photon plans and lower doses to critical structures.
% RBE
% 
% 24Gy(RBE) %http://www.sciencedirect.com/science/article/pii/S0360301615005179
% 
% Hypo-fractionation of liver cancer %http://ro-journal.biomedcentral.com/articles/10.1186/1748-717X-8-59  http://jco.ascopubs.org/content/early/2015/12/11/JCO.2015.64.2710.short
% 
% Future SBRT - 4pi %https://www.aapm.org/meetings/2014SS/documents/FUTUREOFSBRT2014Kupelian.pdf
% 
% 
% Overview of lung cancer %http://www.tandfonline.com/doi/pdf/10.3109/0284186X.2011.590148
% 
% Reoccurent lung cancer %http://www.ncbi.nlm.nih.gov/pubmed/24016675
% 
% 
% Photons vs Protons for stage III NSCLC RTOG 1308
% 
% Stephen Bowen - deliever more dose to damaged lung instead of functionated one
% 
% bio-adaptive particle therapy
% 
% SBRT is the lowest for cost per quality of life for early stage, but Protons beams scanning for advanced %http://www.ncbi.nlm.nih.gov/pubmed/26828647
% SBRT limited: large tumors (>5 cm, more tumors, patients with poor lung funtction)
% 
% Proton SBRT %http://www.ncbi.nlm.nih.gov/pubmed/22551902
% 
% Seattle treated NSCLC with PBS
% 
% ABS vs passive?
% 
% Small tumor are bad %http://www.ncbi.nlm.nih.gov/pmc/articles/PMC4399385/\
% 
% dosimetric study of protons vs sbrt %http://www.ncbi.nlm.nih.gov/pubmed/18405986/

