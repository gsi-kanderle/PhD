\documentclass[type=dr, dr=rernat, accentcolor=tud7b,colorbacktitle, bigchapter, openright, twoside, 12pt ]{tudthesis}
%\documentclass[11pt,twoside,a4paper]{article}
\usepackage[english]{babel} 
\usepackage[utf8]{inputenc}
\usepackage{graphicx}
\usepackage{pstricks}
\usepackage{psfrag}
\usepackage{enumerate}
\usepackage{float}
\usepackage{epsfig}
\usepackage{geometry}
\usepackage{subfigure}
\usepackage{rotating}
\usepackage{minitoc}
% \usepackage{dominitoc}
\usepackage{multirow}
\usepackage{listings}
%\usepackage{appendix}
%\usepackage[breaklinks=true]{hyperref}
%\usepackage{breakcites}

%%%% 1 1/2 facher Zeilenabstand:	
\usepackage{setspace}
\onehalfspacing




\begin{document}
\chapter{Treatment Planning for Complex Patient Geometry}
\label{chapter:vmm}
\minitoc

\section{Introduction}

Lung cancer in late stages has poor prognosis, with lots of metastases.

\section{Materials and Methods}

For creation of treatment plans 4D extension of GSI's treatment planning system TRiP98 \cite{Kraemer2000a, Richter2013} was used and modified. A description of modifications and tools used will be given here, 
alongside with patient data and analysis description.

\subsection{Multiple targets}

TRiP98 optmization works on minimizing residual of a nonlinear equation system \cite{Kraemer2000a}. The cost fuction $E(N)$ for particle number $\vec{N}$ goes as:
\begin{equation}
 E(\vec{N}) = \sum_{i\in target} \left( D_{pre}^{i} - D_{act}^{i}(\vec{N})\right) = \sum_{i\in target} \left( D_{pre}^{i} -\sum_{j=1}^n c_{ij}N\right)
\end{equation}

For a CT voxel $i$, $ D_{pre}$ and $D_{act}$ are the prescribed and actual dose, respectively. The coefficient $c_{ij}$ stands for correlation between dose deposition and field position $j$ at a voxel $i$. The number of fields is denoted with $n$.



\subsection{Patient data}



1, 3, 11, 13, 20, 21, 22, 23

\section{Results}

\section{Summary and Discussion}





\bibliographystyle{apalike}
\bibliography{../ref.bib}{}
% \bibliographystyle{plain}

\end{document}